\documentclass[12pt]{paper}
\usepackage{amsmath}
\usepackage{amssymb}
\usepackage{graphicx}
\usepackage{hyperref}
\usepackage{color}
\usepackage{float}
\begin{document}
\title{Mathematical Definitions}
\author{Ofir Shukron}
\maketitle
\section{P-values}
Given a rejection region $\Gamma$ in the form of $[c,\infty)]$, for the null hypothesis $H_0$,  the p-value of an observed statistics $T=t$ is defined as 
\begin{equation*}
p-value(t)=min_{\{\Gamma;t\in \Gamma\} } \{Pr(T\in\Gamma | H_0\quad true)\}
\end{equation*}
See definition in \cite{storey2002direct}.

\section{Type I and Type II errors}
A type I and type II errors might occur in statistical hypothesis testing when hypothesizing about the observed null hypothesis. Type I error is accepting the alternative (or rejecting the null) when the null is correct, while type II is accepting the null when the alternative is correct. 


\section{Quadratic Forms}\label{section_quadraticForms}
From Mathai \cite{mathai1992quadratic}(Chapters 1 and 4). For a $p\times 1$ random vector $X$ with mean value $E(X)=\mu$ and $Cov(X)=E[(X-E(X))(X-E(X))']=\Sigma>0$, we set 

$ Y=\Sigma^{-0.5}X\Rightarrow E(Y)=\Sigma^{-0.5}\mu$ and $Cov(Y)=\Sigma^{-0.5}Cov(X)\Sigma^{-0.5}=I$
$Z=(Y-\Sigma^{-0.5}\mu)\Rightarrow E(Z)=0$ and $Cov(Z)=I$ 
We can express the quadratic form $Q(X) =X'AX$ with the centralize variable $Z$ and a symmetric positive definite matrix $A$ as
\begin{equation}
Q(X)=X'AX=Y'\Sigma^{0.5}A\Sigma^{0.5}Y = (Z+\Sigma^{-0.5}\mu)'\Sigma^{0.5}A\Sigma^{0.5}(Z+\Sigma^{-0.5}\mu)	
\end{equation}

As an example, if we set $\mu = 0$, $A=I$, $Cov(X)=\Sigma=\sigma I$ and get 
$Q(X)=X'X$, $Y = \frac{X}{\sqrt{\sigma}}$, $Z=(\frac{X}{\sqrt{\sigma}}-0)$, then 
\begin{equation}
Q(X)=X'X= (\frac{X}{\sqrt{\sigma}})'(\sigma I)(\frac{X}{\sqrt{\sigma}})=XX'	
\end{equation}
which is the normal form.

If $P$ is a $p \times p$ orthogonal matrix which diagonalize $\Sigma^{0.5}A\Sigma^{0.5}$, that is 

\begin{equation*}
P'\Sigma^{0.5}A\Sigma^{0.5}P = diag(\lambda_1,\lambda_2,...,\lambda_p)
\end{equation*}
with $P'P=I$, and $\lambda_i$ are the eigenvalues of $\Sigma^{0.5} A \Sigma^{0.5}$.  Then if $U=P'Z$, we have 
\begin{equation*}
Z=PU, \qquad E(U)=0 \qquad Cov(U) = I
\end{equation*}
Then we can express the quadratic form of $X$ by 
\begin{equation}
Q(X) = (U+b)'diag(\lambda_1,\lambda_2,...,\lambda_p)(U+b)
\end{equation}
with $b = (P'\Sigma^{-0.5}\mu)'$ 

\section{Spline interpolation}\label{section_splineInterpolation}
This section is mainly taken from \cite{de1978practical}. The Newton from of the interpolation polynomial is defined using the divided differences as:
\begin{equation*}
p_n(x)=\sum_{i=1}^{n}(x-\tau_1)\cdot\cdot\cdot(x-\tau_n)[\tau_1,\tau_2,..\tau_n]g
\end{equation*}
where $\tau_i$ are the nodes of the function $g$ to be interpolated, in which the polynomial $p_n(x) $ of order $n$ agrees with it i.e $p_n(\tau_i)=g(\tau_i)\quad \forall i=1..n$

\section{the Heat equation $u_t=ku_xx$}
Taken from \cite{bleecker1992basic}. The heat equation is a second order linear parabolic PDE describing the change of temperature in a surface with time. The constant $k$ is called the conductivity . A fundamental source solution to the one-dimensional homogeneous equation is given by 
\begin{equation*}
u(x,t) = (4\pi kt)^{-0.5} \exp(-x^2/(4kt))\quad t>0, -\infty <x <\infty
\end{equation*}
the area under the graph of the solution is always 1. 
In general, to solve the equation we need to specify an initial condition (in terms of time) and two boundary conditions (in terms of space).


To begin to solve the heat equation, we search for product solutions of the form $u(x,t)=X(x)T(t)$, thus we perform the differentiation to obtain a function of time to equal a function of space, which is only possible if the two equal a constant $c$. Thus we arrive at two equations
\begin{equation*}
T'(t)-kcT(t)=0
\end{equation*}
\begin{equation*}
X''(x)-cX(x)=0
\end{equation*}
which their solution depends on the value of $c$
\begin{equation*}
u(x,t)=e^{-\lambda^2kt}(c_1\sin(\lambda x)+c_2\cos(\lambda x) ,\quad c=-\lambda^2<0
\end{equation*}
\begin{equation*}
u(x,t)=e^{\lambda^2 kt}(c_1e^{\lambda x}+c_2e^{-\lambda x}),\quad c=\lambda^2 >0
\end{equation*}
\begin{equation*}
u(x,t)=c_1x+c_2, \quad c=0
\end{equation*}

If $u(x,t)$ is a solution of $u_t = ku_{xx}$ then
\begin{enumerate}
\item  $v(x,t)=u(ax-x_0,a^2t-t_0)$ is a solution of $v_t=kv_{xx}$ for any $a,x_0,t_0$
\item $v(x,t)=u(x,(k'/k)t)$ is a solution of $v_t=k'v_{xx}$
\end{enumerate}

The solutions of the heat equations are \textbf{unique}, in the sense that if $u_1,u_2$ are solutions of $u_t=ku_{xx}$ with $0\leq x\leq L, t\geq 0$, $u(0,t)=a(t), u(L,t)= b(t), u(x,0) = f(x)$, then $u_1=u_2$ 

The stability, or change of solution in respect to small variation in the initial conditions, is seen through the \textbf{maximum principle}. If $u(x,t)$ is a $C^2$ solution of $u_t=ku_{xx}$ with $0\leq x\leq L, t\geq 0$, $u(0,t)=a(t), u(L,t) = b(t), u(x,0) = f(x)$, with $a,b,f$ a given $C^2$ functions with $A,B,M$ the maximum temperature of $a(t),b(t),f(x)$ respectively and $\bar{M}=max(A,B,M)$. Then $u(x,t)\leq \bar{M}$ in the whole range. 
The proof of the maximal principle can be found in \cite{bleecker1992basic} page 148.

For an \textbf{inhomogeneous heat equation}, with a heat source term $Q(x,t)$ we have 
\begin{equation*}
u_t=ku_{xx}+Q(x,t)
\end{equation*}
with B.C $u(-1,t)=u(1,t)$ and I.C $u(x,0)=f(x)=const$, we perform the following steps:
\begin{enumerate}
\item we first solve the homogeneous equation $u_t=ku_{xx}$ by separation of variables, to find the eigenfunctions $\sin(n\pi x)$ and eigenvalues $(n\pi)^2$, $n=1..\infty$
\item then the solution can be expanded in terms of the eigenfunctions as 
\begin{equation*}
u=\sum_{n=1}^{\infty}c_n(t)\sin(n\pi x)
\end{equation*}
\item assume that the source term can be expanded as a Fourier series
\begin{equation*}
Q(x,t)= \sum_{n=1}^{\infty}d_n\sin(n\pi x)
\end{equation*}
From the Fourier series theory we know that 
\begin{equation*}
d_n=\int Q(x,t)\sin(n\pi x)dx
\end{equation*}
but $d_n$ must be a function of time and only a constant as a function of $x$, therefore setting $a_n(t) = d_n$ and assuming we can perform the integration 
\begin{equation*}
Q(x,t) = \sum_{n=1}^{\infty}a_n(t)\sin(n\pi x)
\end{equation*}
\item we now substitute the series expansion of $u$ into the inhomogeneous equation to obtain 
\begin{equation*}
\sum_{n=1}^{\infty}c'_n(t)\sin(n\pi x)=-k\sum_{n=1}^{\infty}(n\pi)^2c_n(t)\sin(n\pi x)+\sum_{n=1}^{\infty}a_n(t)\sin(n\pi x)
\end{equation*}
\item the Fourier coefficients must be equal, then we have the ODE
\begin{equation*}
c'_n(t)= -k(n\pi)^2 c_n(t)+a_n(t)=-k\lambda_n c_n(t)+a_n(t)
\end{equation*}
\item we can solve the ODE by multiplying by the integration factor $e^{\lambda_n k t}$
\end{enumerate}

The \textbf{Duhanel's principle} for inhomogeneous heat equations


% The bibliography
\bibliographystyle{plain}
\bibliography{MathematicalDefinitionsBibliography} % the bibliography.bib file 
\end{document}