\documentclass[12pt]{paper}
\usepackage{amsmath}
\usepackage{amssymb}
\usepackage{graphicx}
\usepackage{hyperref}
\usepackage{color}
\usepackage{float}
\begin{document}
\title{Mathematical Definitions}
\author{Ofir Shukron}
\maketitle
\section{P-values}
Given a rejection region $\Gamma$ in the form of $[c,\infty)]$, for the null hypothesis $H_0$,  the p-value of an observed statistics $T=t$ is defined as 
\begin{equation*}
p-value(t)=min_{\{\Gamma;t\in \Gamma\} } \{Pr(T\in\Gamma | H_0\quad true)\}
\end{equation*}
See definition in \cite{storey2002direct}.

\section{Type I and Type II errors}
A type I and type II errors might occur in statistical hypothesis testing when hypothesizing about the observed null hypothesis. Type I error is accepting the alternative (or rejecting the null) when the null is correct, while type II is accepting the null when the alternative is correct. 


\section{Quadratic Forms}\label{section_quadraticForms}
From Mathai \cite{mathai1992quadratic}(Chapters 1 and 4). For a $p\times 1$ random vector $X$ with mean value $E(X)=\mu$ and $Cov(X)=E[(X-E(X))(X-E(X))']=\Sigma>0$, we set 

$ Y=\Sigma^{-0.5}X\Rightarrow E(Y)=\Sigma^{-0.5}\mu$ and $Cov(Y)=\Sigma^{-0.5}Cov(X)\Sigma^{-0.5}=I$
$Z=(Y-\Sigma^{-0.5}\mu)\Rightarrow E(Z)=0$ and $Cov(Z)=I$ 
We can express the quadratic form $Q(X) =X'AX$ with the centralize variable $Z$ and a symmetric positive definite matrix $A$ as
\begin{equation}
Q(X)=X'AX=Y'\Sigma^{0.5}A\Sigma^{0.5}Y = (Z+\Sigma^{-0.5}\mu)'\Sigma^{0.5}A\Sigma^{0.5}(Z+\Sigma^{-0.5}\mu)	
\end{equation}

As an example, if we set $\mu = 0$, $A=I$, $Cov(X)=\Sigma=\sigma I$ and get 
$Q(X)=X'X$, $Y = \frac{X}{\sqrt{\sigma}}$, $Z=(\frac{X}{\sqrt{\sigma}}-0)$, then 
\begin{equation}
Q(X)=X'X= (\frac{X}{\sqrt{\sigma}})'(\sigma I)(\frac{X}{\sqrt{\sigma}})=XX'	
\end{equation}
which is the normal form.

If $P$ is a $p \times p$ orthogonal matrix which diagonalize $\Sigma^{0.5}A\Sigma^{0.5}$, that is 

\begin{equation*}
P'\Sigma^{0.5}A\Sigma^{0.5}P = diag(\lambda_1,\lambda_2,...,\lambda_p)
\end{equation*}
with $P'P=I$, and $\lambda_i$ are the eigenvalues of $\Sigma^{0.5} A \Sigma^{0.5}$.  Then if $U=P'Z$, we have 
\begin{equation*}
Z=PU, \qquad E(U)=0 \qquad Cov(U) = I
\end{equation*}
Then we can express the quadratic form of $X$ by 
\begin{equation}
Q(X) = (U+b)'diag(\lambda_1,\lambda_2,...,\lambda_p)(U+b)
\end{equation}
with $b = (P'\Sigma^{-0.5}\mu)'$ 

% The bibliography
\bibliographystyle{plain}
\bibliography{MathematicalDefinitionsBibliography} % the bibliography.bib file 
\end{document}