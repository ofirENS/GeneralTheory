\documentclass[12pt]{report}
\usepackage{amsmath}
\usepackage{amssymb}

\begin{document}
\tableofcontents
\chapter{General Probability Theory}
\section{$\sigma$-algebra}
From \cite{Oksendal14}. If $\Omega$ is a given set, the a $\sigma$-algebra $\mathcal{F}$ on $\Omega$ is a family $\mathcal{F}$ of subsets of $\Omega$ with the following properties
\begin{enumerate}
\item[(i)]$\emptyset\in\mathcal{F}$
\item[(ii)] if a set $F\in\mathcal{F}$ then $F^C \in\mathcal{F}$, where $F^C =\Omega\backslash F$ 
\end{enumerate}


\chapter{}

\section{Noise}[Unfinished]
White noise is a process for which there is no correlation between any values at different times. As an idealization of such a process there exist the Orenstein-Uhlenbeck process (to be defined) for which produces an almost uncorrelated noise. For this model, the second order correlation function can, up to a constant factor, be described as 
\begin{equation*}
<x(t),x(t')>= \frac{\gamma}{2}\exp(-\gamma|t-t'|)
\end{equation*}

\section{Cylinder Sets}
A cylinder set of Brownian trajectories is defined by a sequence of times $0\leq t_1<t_2<...<t_n$, and real intervals $I_k=(a_k,b_k)$, $k=1...n$, as 
\begin{equation*}
C(t_1,t_2,...,t_n;I_1,I_2,...I_n)=\{\omega\in \Omega|x(\omega,t_k)\in I_k\},\forall 1\leq k \leq n
\end{equation*}
For a cylinder not to contain a trajectory, it is enough that in one time point the trajectory is not in the interval.
The cylinder $C$ contains entire trajectory, which fulfill the inclusion demand. 

The joint PDF of $w(t_1,\omega),w(t_2,\omega),w(t_3,\omega)$ is the Weiner measure of the cylinder $C(t_1,t_2,t_3;I^x,I^y,I^z)$. These points belong to the same process $\omega$ at three different times. 

\section{Filtration}
The $\sigma$-algebra $F_t$ of Brownian event is defined by cylinder
sets confined to times $0\leq t_i < t$, for some fixed $t$. Obviously, $F_s \subset F_t \subset F$ if $0\leq s < t <\infty$. The family of $\sigma$− algebras $F_t$ for $t \geq 0$ is called the Brownian filtration and is said to be generated by the Brownian events up to time $t$

In simple words, filtration is an increasing sequence of $\sigma$-algebras on a measurable space. 

\section{Adapted Process}
The process $x(t,\omega)$ is said to be adapted
to the Brownian filtration $F_t$ if $\{\omega\in\Omega|x(t,\omega) \leq y\} \in F_t$ for every $t \geq 0$ and
$y\in R$. In that case we also say that $x(t, \omega)$ is $F_t$-measurable.

Thus an adapted process does not depend on the future behavior of the Brownian trajectory from time $t$ on.

\section{The Weiner Measure}\label{theWeinerMeasure}
The probability measure $Pr$ defined on all the Brownian trajectories (events in the set $\Omega$) are defined on cylinder sets and then extended to all events in the space by elementary properties of probability sets.

For a cylinder set $C(t;I)$, where $t>0$ and $I=(a,b)$, the Weiner measure is 
\begin{equation*}
Pr\{C(t;I)\}=\frac{1}{\sqrt{2\pi t}}\int_a^b\exp^{-x^2/2t}dx
\end{equation*}

\subsection{Martingale}
A martingale is a stochastic process $ x(t)$ with $\mathbb{E}|x(t)|<\infty$ ,$\forall t$, and for $t_1<t_2<...<t_n$
\begin{equation*}
\mathbb{E}[x(t)|x(t_1)=x1,x(t_2)=x_2,...x(t_n)=x_n]=x_n
\end{equation*}

\section{The Langevin Equation}{[Unfinished]}
An equation of the type 
\begin{equation*}
\frac{dx}{dt}=a(x,t)+b(x,t)\xi(t)
\end{equation*}

\section{The It\^{o} Equation}
Equation of the form 
\begin{equation*}
 \mathrm{d} X_t = \mu(X_t,t)\, \mathrm{d} t +  \sigma(X_t,t)\, \mathrm{d} B_t
\end{equation*}
is an informal way of expressing the more appropriate integral equation 
\begin{equation*}
X_{t+s} - X_{t} = \int_t^{t+s} \mu(X_u,u) \mathrm{d} u + \int_t^{t+s} \sigma(X_u,u)\, \mathrm{d} B_u
\end{equation*}
where the dynamic of the particle $X$ is given as a sum of two integrals, the first one is a standard Lebesgue integral, and the second is an It\^{o} integral (to be defined).  
The functions $\mu(X_t,t)$ and $\sigma(X_t,t)$ are well defined continuous functions, and $B_t$ is a Brownian motion (a Weiner process, or white noise). An informal way of interpreting the equation above is to say that at each small time interval of size $\delta$ the process $X_t$ is changed by a normally distributed value with expectation $\mu(X_t,t)\delta$ and variance $\sigma(X_t,t)^2\delta$. 
the function $\mu(X_t,t)$ is referred to a s the drift coefficient, and the function $\sigma(X_t,t)$ as the diffusion coefficient. 

\section{Existence and Uniqueness of It\^{o} SDE solutions}
For an Ito SDE taking values in n-dimensional Euclidean space, if for $T>0$\\
\begin{eqnarray*}
&\mu:&\mathbb{R}^2\times[0,T]\rightarrow \mathbb{R}^n\\
&\sigma:&\mathbb{R}^n \times[0,T] \rightarrow \mathbb{R}^{n\times m}
\end{eqnarray*}
are measurable functions, for which there exist constants $C$ and $D$ such that 
\begin{eqnarray*}
&&|\mu(x,t)|+|\sigma(x,t)|\leq C(1+|x|)\\
&&|\mu(x,t)-\mu(y,t)|+|\sigma(x,t)-\sigma(y,t)|\leq D|x-y|
\end{eqnarray*}
for all $t\in [0,T]$ and all $x,y\in \mathbb{R}^n$, and $|\sigma|^2=\sum_{i,j=1}|\sigma_{i,j}|^2$. 
If $Z$ is a random variable, independent of the $\sigma$-algebra generated by $B_s$, $s\geq 0$, and with finite second moment, then the SDE, with initial condition $X_0=Z$ has an almost surely unique solution in $t\in[0,T]$, $X_t(\omega)$, such that $X$ is adapted to the filtration $F_t$ generated by $Z$ and $B_s$, $s\leq t$.


\section{Stochastic Integration}
For an arbitrary function of time $G(t')$ and a Weiner process $W(t)$, the stochastic integral 
\begin{equation*}
\int_{t_0}^t G(t')\mathrm{d}W(t')
\end{equation*}
is defined as a kind of Riemann integral, that is, we divide the interval $[t_0,t]$ into $n$ subintervals $t_0\leq t_1\leq ...\leq t$, and define intermediate points $t_{i-1}\leq\tau_i\leq t_{i}$. The integral is then defined as a limit of the partial sums 
\begin{equation*}
S_n=\sum_{i=1}^nG(\tau_i)[W(t_i)-W(t_{i-1})]
\end{equation*}
Note that the Weiner process in the partial sums is contributing the values at the edges of the interval, whereas the function $G$ contributes the values somewhere in its middle.
The choice of $\tau_i$ affects the value of the integral. It can be seen that f we choose $\tau_i=\alpha t_{i-1}+(1-\alpha)t_i$, with $0\leq \alpha\leq 1$,  than the mean value of the integral can take any value between 0 and $(t-t_0)$. 

for the \textbf{It\^{o} integral}, we choose $\tau_i=t_{i-1}$, which is the left boundary of the subinterval and thus define the stochastic integral as
\begin{equation*}
\int_{t_0}^{t}G(t')\mathrm{d}W(t')=ms-lim_{n\rightarrow\infty}\left\{\sum_{i=1}^nG(t_{i-1})[W(t_{i})-W(t_{i-1})] \right\}
\end{equation*}
by ms-lim, we mean the mean square limit, which is defined by 
\begin{equation*}
ms-lim_{n\rightarrow\infty}\int p(\omega)[X_n(\omega)-X(\omega)]^2dw= lim_{n\rightarrow \infty}<(X_n-X)^2>
\end{equation*}

if indeed $lim_{n\rightarrow \infty}<(X_n-X)^2>$ we say that $ms-lim_{n\rightarrow \infty} X_n=X$.

One major consequence of the choice $\tau_i=t_{i-1}$ is that 
\begin{equation*}
\int_{t_0}^{t}W(t')dW(t')=0
\end{equation*}
That is, the stochastic integral of the Weiner process over any interval is zero.
Because the left point of the subinterval is chosen, the value of the function $G(t'_{i-1})$ is independent of the increments $[W(t_i)-W(t_{i-1})]$, since $G$ is $F_t$ adapted process.
\section{Properties of It\^{o} integral}

We define the class $H_2[0,T]$ of $F_t$-adapted stochastic processes such that 
\begin{equation*}
\int_0^T \mathbb{E}f^2(s,\omega)ds <\infty
\end{equation*}
and list the following properties
\begin{enumerate}
\item \textbf{linearity}. For $f(t),g(t)\in H_2[0,T]$, and $\alpha,\beta$ real, then $\alpha f(t)+\beta g(t) \in H_2[0,T]$, and 
\begin{equation*}
\int_0^t \alpha f(s)+\beta g(s) d\omega(s) = \alpha \int_0^t f(s)d\omega(s) +\beta \int_0^t g(s)d\omega(s)
\end{equation*}
\item \textbf{additivity}. If $f(t)\in H_2[0,T_1]$ and $f(t)\in H_2[T_1,T]$ for $0<T_1<T$, then 
\begin{equation*}
\int_0^Tf(s)d\omega(s) = \int_0^{T_1}f(s)d\omega(s) +\int_{T_1}^T f(s)d\omega(s)
\end{equation*}
\item for a deterministic and integrable $f(t)$
\begin{equation*}
\int_0^t f(s)d\omega(s) \sim N\left(0,\int_0^t f^2(s)ds\right)
\end{equation*}
\item for $f(t)\in H_2[0,T]$
\begin{equation*}
\mathbb{E}\int_0^t f(s)d\omega(s) = 0
\end{equation*}
\item for $0<\tau<t<T$  
\begin{equation*}
\mathbb{E}\left[\int_0^t f(s)d\omega(s)  | \int_0^\tau f(s)d\omega(s)=x \right] =x 
\end{equation*}
\item for $f(t),g(t)\in H_2[0,T]$
\begin{equation*}
\mathbb{E}\left[\int_0^Tf(s)d\omega(s)\int_0^T g(s)d\omega(s)\right]=\int_0^T\mathbb{E}[f(s)g(s)]ds
\end{equation*}
note that the right-hand-side is integrated with respect to $s$
\end{enumerate}
\section{Non-anticipating Functions}{[Unfinished]}

\section{Stochastic Differential Equations}



% The bibliography
\bibliographystyle{plain}
\bibliography{stochasticProcessesBibliography} % the bibliography.bib file 
\end{document}