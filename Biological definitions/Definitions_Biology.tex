\documentclass[12pt]{paper}
\usepackage{amsmath}
\usepackage{amssymb}
\usepackage{graphicx}
\usepackage{hyperref}
\usepackage{color}
\begin{document}
\title{Definitions-Biology}
\maketitle
\section{Histones}
...

Hyperaccetlation of the histones leads to unfolding of the chromatin that should facilitate the general accessibility of factors to the DNA \cite{Blackwood98}.

\section{Enhancers}
 An \href{http://en.wikipedia.org/wiki/Enhancer_(genetics)}{Enhancer} is a short (20-400bp,\cite{Kulaeva12}) region of DNA that can be bound with proteins to activate transcription of a gene or genes. These proteins are usually referred to as transcription factors. Enhancers are discrete DNA elements that contain  specific sequence motifs with which DNA-binding proteins interact and transmit molecular signals to genes \cite{Blackwood98}.

Enhancers increase transcription of genes in a manner that is independent of their orientation and distance relative to the RNA start site \cite{Blackwood98} 


\section{Promoters}
Promoters (or core promoters) are located within $\pm 40$ nucleotides from the RNA start site \cite{Blackwood98}.

\section{Insulators}

\section{Promoter-Enhancer Interactions}
In the majority of cases, action of enhancers involves enhancer-promoter interaction through proteins bound at the enhancer and promoter, accompanied by formation of an intervening chromatin loop \cite{Kulaeva12}.

There are two mechanisms by which enhancer-promoter  selectivity might be achieved. First, there could be specific interactions between enhancer -binding proteins and factors that interact with the promoter. Second, boundary elements (insulators) could be used to block undesired enhancer-promoter interactions \cite{Blackwood98}. 

How might enhancer binding proteins and their associated co-activators  establish productive interaction with the promoter? One option is the DNA looping. A "facilitated tracking" mechanism for enhancer function is postulated to allow enhancer bound complex containing DNA-binding factors and co-activators "tracks" via small steps along the chromatin until it encounters the promoter, at which a stable looped structure is formed \cite{Blackwood98}. After formation of the activation chromatin loop, it is likely stabilized by additional protein factors, such as CTCF and cohesin \cite{Kulaeva12}. CTCF is an insulator DNA binding protein. 


It was shown that formation of chromatin loop topologically isolating the enhancer from the target promoter is sufficient to block enhancer-promoter communication. 
% The bibliography
\bibliographystyle{plain}
\bibliography{thebibliography} % the bibliography.bib file 
\end{document}