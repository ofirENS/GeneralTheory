\documentclass[12pt]{paper}
\usepackage{amsmath}
\usepackage{amssymb}
\usepackage{graphicx}
\usepackage{hyperref}
\usepackage{color}
\begin{document}
\title{Definitions-Biology}
\maketitle
\section{Enhancers}
 An \href{http://en.wikipedia.org/wiki/Enhancer_(genetics)}{Enhancer} is a short (50-150bp) region of DNA that can be bound with proteins to activate transcription of a gene or genes. These proteins are usually referred to as transcription factors. Enhancers are discrete DNA elements that contain  specific sequence motifs with which DNA-binding proteins interact and transmit molecular signals to genes \cite{Blackwood98}.

Enhancers increase transcription of genes in a manner that is independent of their orientation and distance relative to the RNA start site \cite{Blackwood98} 


\section{Promoters}
Promoters (or core promoters) are located within $\pm 40$ nucleotides from the RNA start site \cite{Blackwood98}.

\section{Insulators}

\section{Promoter-Enhancer Interactions}
There are two mechanisms by which enhancer-promoter  selectivity might be achieved. First, there could be specific interactions between enhancer -binding proteins and factors that interact with the promoter. Second, boundary elements (insulators) could be used to block undesired enhancer-promoter interactions \cite{Blackwood98}. 

% The bibliography
\bibliographystyle{plain}
\bibliography{thebibliography} % the bibliography.bib file 
\end{document}