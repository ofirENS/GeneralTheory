\documentclass{paper}
\usepackage{amsmath}
\usepackage{amssymb}
\usepackage{graphicx}
\usepackage{hyperref}
\usepackage{color}
\begin{document}
\title{Theory of Polymer Dynamics}
\maketitle
\section{Introduction}
This document summarizes information related to the construction of theoretical models representing polymers. 
It is mainly drawn from the book by Doi \cite{doi1986theory}. 
 
\section{The freely jointed chain}\label{section_theFreelyJointedChain}
From \cite{doi1986theory}.We start with a simple model: a chain consisting of $N$ links, each of length $b_0$ and able to point in any direction independently of each other. The conformation of the model is represented by $(N+1)$ position vectors $\{R_n\}=\{R_0,R_1,...,R_{N}\}$, or by the set of bond vectors $\{r_n\}=(r_1,r_2,...,r_N)$, where $r_n=R_{n}-R_{n-1}$. 

Since the bonds are independent, the configuration distribution is defined as 
\begin{equation*} 
\Psi(\{\vec{r}_n\})=\prod_{n=1}^N\psi(\vec{r}_n)
\end{equation*}
for vector of constant length $b_0$
\begin{equation*}
\psi(\vec{r})=\frac{1}{4\pi b_0^2}\delta(|\vec{r}|-b_0)
\end{equation*}
we then normalize such that 
\begin{equation*}
\int \psi(\vec{r})d\vec{r} =1
\end{equation*}
If the bond vectors are presented using spherical coordinates with angles $\theta$ and $\phi$, then the configuration distribution of the chain is given by 
\begin{equation*}
\psi(\theta^{N},\phi^{N})=\left(\frac{1}{4\pi}\right)^{N} \prod_{i=1}^{N}\sin(\theta_i)
\end{equation*}
This is due to the fact that the surface element of the sphere is given by the determinant of the Jacobi matrix, $J$, of spherical transformation for each $R_i=(x,y,z)$, namely
\begin{eqnarray*}
x &=& b_0\sin(\theta)\cos(\phi)\\
y &=& b_0\sin(\theta)\sin(\phi)\\
z &=& b_0\cos(\theta) 
\end{eqnarray*} 
which is $|J|=b_0^2\sin(\theta)$, to get the probability that a bond vector is in the range $d\theta d\phi$, we divide $|J|$ by the surface area of the sphere which is $4\pi b_0^2$ to get $\sin(\theta)/4\pi$ for each bond vector. Since the bond vectors are independent, we get the multiplicative expression for $\psi$.
 
Any average quantity $B(\theta,\phi)$ that depends on the configuration of the chain in equilibrium will be written as  
\begin{equation*}
\left< B\right>=\int\int B \psi d\theta^{N}d\phi^{N}
\end{equation*}
The above property of expected value of a quantity $g(x)$ given that we know only the distribution function $f(x)$ is the called the Law of the unconscious statistician.
\begin{equation*}
E[g(x)]=\int_{-\infty}^{\infty}g(x)f(x)dx = \int_{-\infty}^{\infty}g(x)dF(x)dx
\end{equation*} 
were $F(x)$ is the cdf (if known).

\subsection{The end-to-end vector}\label{subsection_theEndToEndVector}
We define the end-to-end vector as $\vec{R}=\vec{R}_N-\vec{R}_0=\sum_{n=1}^N \vec{r}_n$. The quantity $\left<R^2\right>$ characterizes the length of the chain. The standard deviation of the end-to-end vector
\begin{equation*}
\bar{R}=\left<R^2\right>^{0.5}=\left<(R_N-R_0)^2\right>^{0.5}=\sqrt{N}b_0
\end{equation*} 
since 
\begin{equation*}
\left<R^2\right> = \left<\left(\sum_{n=1}^N(r_n)\right)^2\right>=\sum_{n=1}^N\sum_{m=1}^N\left<r_nr_m\right>+\sum_{n=1}^N\left<r_n^2\right>+2\sum_{m>n}^N\left< r_nr_m\right>=\sum_{n=1}^N\left<r_n^2\right>=Nb_0^2
\end{equation*}

therefore, $\left<R^2\right>^{0.5}=\sqrt{N}b_0$. Here we used  $\left<r_nr_m\right>=\left<r_n\right>\left<r_m\right>=0$. Note that this is just the sum of variances of the bond vectors (each of which has variance of $b_0$)

\section{Distribution of the end-to-end vector}\label{section_distributionOfTheEndToEndVector}
We denote $\Phi(R,N)$ as the probability distribution function that the end-to-end vector of a chain of N beads is $R$.
\begin{equation*}
\Phi(R,N) =\int\int\int...\int\delta(R-\sum_{n=1}^Nr_n)\Psi(\{r_n\}))dr_1dr_2...dr_N
\end{equation*}
in short, we take all chain configurations for which the end-to-end vector is exactly $R$. Using the Fourier transform identity for the delta function
\begin{equation*}
\delta(r)= \frac{1}{(2\pi)^3}\int e^{ikr} dk
\end{equation*}
see \href{http://en.wikipedia.org/wiki/Dirac_delta_function}{wikipedia}.
where, after some manipulation and using polar coordinates for solving the integral, we get
\begin{equation*}
\Phi(R,N)=\frac{1}{(2\pi)^3}\int e^{ikR} \left(\frac{\sin(kb)}{kb}\right)^N dk
\end{equation*}
Continuing with a series of approximations to the integral above, we get
\begin{equation*}
\Phi(R,N)=(3/2\pi N b^2)^{1.5} \exp(-\frac{3R^2}{2Nb^2})
\end{equation*}


\section{The Gaussian Chain}\label{section_theGaussianChain}
From \cite{doi1986theory}. We consider a chain whose bond length has a Gaussian distribution 
\begin{equation*}
\psi(r)=\left[\frac{3}{2\pi b^2} \right]^{3/2} \exp \left( -\frac{3r^2}{2b^2} \right)
\end{equation*}
in 3D,  with bond-length variance $<r^2> =b^2$. The Gaussian chain represent the model of beads on a strings, with the harmonic potential energy 
\begin{equation*}
U_0(\{R_n\})\frac{3k_BT}{2b^2}\sum_{n=1}^N (R_n-R_{n-1})^2
\end{equation*}

Since the sum of two Gaussian random variables is again a random variable, it turns out that the distribution of the bond between any two beads $m$ and $n$ is Gaussian with 
\begin{equation*}
\Phi(R_n-R_m,n-m)=\left[\frac{3}{2\pi b^2 |n-m|}\right]^{3/2}\exp \left[-\frac{3(R_n-R_m)^2}{2|n-m|b^2}\right]
\end{equation*}
and for any $m$ and $n$ 
\begin{equation*}
<(R_n-R_m)^2> = |n-m|b^2
\end{equation*}

From the formula of the distribution, we can immediately see that setting $R_N=R_m$, meaning placing two beads at the same position, gives us the encounter probability at steady state, namely $A|n-m|^(3/2)$, with $A$ some constant of normalization. 

We can thus immediately see that when setting $R_n=R_m$ we get 
\begin{equation*}
\Phi(0,|n-m|)=(3/2\pi b^2 |n-m|)^{1.5}
\end{equation*}
which is the steady state probability of encounter, which drops with the distance

\section{The center of mass}\label{section_theCenterOfMass}

The center of mass is defined as the mean bead position over time 

\begin{equation*}
cm(t) = \frac{1}{N+1}\sum_{i=0}^N{R_n(t)}
\end{equation*}

therefore, the differential equation for the center of mass is 

\begin{equation*}
\frac{dcm}{dt}=\frac{1}{N+1}\sum_{i=0}^N{\frac{dR_i}{dt}}= \frac{1}{N+1}\sum_{i=0}^Nf_n(t)
\end{equation*}

\section{The source of the Hookian behavior}
From \cite{bird1987dynamics}. Let us look at the probability density function of the end-to-end vector being length $r$ at equilibrium
\begin{equation*}
P_{eq}(r)= \left(\frac{3}{2\pi (N-1)b^2}\right)\exp(-3r^2/2(N-1)b^2)
\end{equation*}

 The Helmholtz free energy $A=U-TS$, with $U$ the internal energy and $S$ the entropy, of the system in equilibrium at temperature $T$ is given by 
\begin{equation*}
A=-KT\ln(Z)
\end{equation*}
with $Z$ the partition function, which is defined as 
\begin{equation*}
Z=c\int e^{(\mathbb{K+\phi})/(KT)}
\end{equation*}
with $c$ some constant,$\mathbb{K}$ is the kinetic energy of the system, $\mathbb{\phi}$ is the potential energy of the system, over the phase space of the system. 

When the length of the chain is changed by a small amount, the change in the Helmholtz free energy is 
\begin{equation*}
dA=-KTd\ln{P_{eq}}=\frac{3KT}{(N-1)b^2}(r\cdot dr)
\end{equation*}

We also know that the change in the Helmholtz free energy is related to the tension $F^c$ of the chain by 
\begin{equation*}
dA=(F^c\cdot dr)
\end{equation*}
equating these two quantities we get 
\begin{equation*}
F^c(r) = -\frac{3KT}{(N-1)b^2}r	 \qquad N-large
\end{equation*}
therefore, we see that the tension behaves as a Hookian spring. If we set $N=2$ we get the spring constant between any two beads.

\section{Centralizing the chain}\label{section_centralizingTheChain}

The differential equation describing the dynamics $R_n$ in $3D$ is

\begin{equation*}
\frac{dR_n}{dt} = -\frac{3D}{b^2}\left(2R_n(t)-R_{n-1}(t)-R_{n+1}(t) \right)+f_n(t)
\end{equation*}

If we shift the chain so that its new center of mass always lays at the origin, we get 

 \begin{equation*}
\frac{d(R_n-cm)}{dt} = -\frac{3D}{b^2}\left(2R_n-R_{n-1}-R_{n+1} \right)+f_n-\frac{1}{N+1}\sum_{i=0}^Nf_i
\end{equation*}

for such a new system, the new center of mass, $cm^*$ should have zero derivative, indeed. 
\begin{equation*}
\frac{dcm^*}{dt}=\frac{1}{N+1}\left[ \sum_{i=1}^N\frac{dR_i}{dt} - (N+1)\sum_{i=0}^N\frac{1}{N+1}f_i \right]= \vec{0}
\end{equation*}
Of course, this process is only valid post simulation. It seems not plausible to simulate a random process such that its center of mass remains in one point. 

Subtracting the center of mass from the position of each bead should not affect the distribution of bond lengths, and it is indeed obvious to see it by subtracting two consecutive beads equation that the center of mass cancels out. However, the distribution of bead position is affected. 

since each $f_n$ is normally distributed with std $\sqrt{2D}$ and mean 0, the random variable $\frac{1}{N}\sum_{i\neq n}^Nf_n(t)$ 

[UNFINISHED]

\section{The distribution of the bond lengths}\label{section_distributionOfTheBondLength}
The bond length behaves like the central $\chi ^2$ with $d$ degrees of freedom, where $d$ is the dimension of the problem. 
In $3D$, since each $\vec{r}_i=[x_i,y_i,z_i]$ has 3 coordinates, each i.i.d normally distributed with $\left<x_i^2\right>=\left<y_i^2\right>=\left<z_i^2\right>=\frac{b^2}{3}$, the length squared $\|\vec{r}_i\|^2=x_i^2+y_i^2+z_i^2\sim\chi^2(3)$ 
with variance $\mu=b^2/3+b^2/3+b^2/3=b^2$ which is the sum of variances.

To calculate the cumulative distribution function (CDF) of the bond length, we use result from the analysis of quadratic forms \cite{mathai1992quadratic} (chapter 3). From the CDF we will obtain the \textit{survival probability}, which its first moment is the \textit{mean first passage time} for the two ends of the polymer.

For the polymer chain, the bond length is a random variable, $b\sim \mathcal{N}(0,\sigma_b)$


\section{The dynamics of bond length}\label{section_theDynamicsOfBondLength}
In simulations we would like to determine $\Delta t$ such that no 'blow-ups' will occur. For this end, we constrain the mean of squared difference between bond length in consecutive steps to lay below some predefined small value. 

The result show that 
\begin{equation*}
\gamma=\ll\|r_n(t+\Delta t)-r_n(t) \|^2\gg = \left(\frac{3D\Delta t}{b} \right)^2+6D\Delta t
\end{equation*}

[UNFINISHED] 
\section{The Rouse Chain}\label{section_theRouseChain}
From \cite{doi1986theory} Chap. 4. The Rouse model was suggested as the basis for polymer dynamics in dilute solution. Let $\{R\}=\{R_1,R_2,...,R_N\}$ be the coordinates of the beads of the chain. The motion of the beads is described using the Smoluchowski equation
\begin{equation*}
\frac{\partial \Psi}{\partial t }=\sum_n \frac{\partial}{\partial R_n}H_{nm}\left[k_BT\frac{\partial \Psi}{\partial R_m}+\frac{\partial U}{\partial R_m}\Psi\}\right]
\end{equation*}


\section{Mean First End To end Encounter Time Of the Rouse Chain}\label{sectioni_meanEndToEndEncounterTimeOfTheRouseChain}
From \cite{amitai2012computation}. 

\section{Eigenvalues of the Rouse Ring}\label{section_eigenvaluesOfTheRouseRing}
Connecting bead 1 and $N$ to form a ring, we now search for the eigenvalues of the new \textit{Rouse ring} matrix, $R$, harboring such a connection. For this end, we have to calculate the determinant of the matrix
\begin{equation*}
D_N=|R-\lambda I|=\left|
\begin{matrix}
 x  & -1 &  0 &  0 &...&   & -1 \\
-1  &  x & -1 &  0 &...&   &  0 \\
 0  & -1 &  x & -1 &...&   &  0 \\
 .  &    &    &  . &   &   &  . \\
  . &   .&    &    &  .&   & \\
 .  &    &    & -1 & x &-1 & 0 \\
 0  &    &    &    & -1& x & -1 \\
 -1 &   .&  . & .  &  0&-1 &  x \\     
\end{matrix}
\right|_{N\times N}
\end{equation*}

with $x=2-\lambda$. We can apply to the matrix of size $(N-1)\times (N-1)$ the same procedure as before and try to solve it recursively. The boundary conditions now read
\begin{equation*}
D_1 = x; D_2 = x^2-1
\end{equation*}
According to the solution in section \ref{section_eigenvaluesOfTheRouseMatrix} and in \cite{lin2011polymer}, we have 
\begin{equation*}
D_{N-1}= \frac{\sin(N\theta)}{\sin(\theta)}
\end{equation*}

The relationship between $D_N$ and $D_{N-1}$ is found by developing the determinant of $D_N$ according to the last column
\begin{equation*}
D_N = x(-1)^{2N}D_{N-1}+(-1)(-1)^{2N-1}D^*+(-1)(-1)^{N+1}D^{**}=xD_{N-1}+D^*+(-1)^{N+2}D^{**}
\end{equation*}
with the determinant $D^*$ and $D^{**}$ defined as the minors 
\begin{equation*}
D^*= 
\left| \begin{matrix}
 x  & -1 &  0 &  0  &... &  0 \\
-1  &  x & -1 &  0  &... &  0 \\
 0  & -1 &  x & -1  &... &    \\
 .  &    &    &  .  &    &    \\
 .  &    &    &  .  &    &    \\
 .  &    &    &  -1 &  x &-1  \\
 -1 &    &    &     &  0 &-1  \\   
\end{matrix}\right|_{(N-1)\times(N-1)} 
D^{**}=\left|\begin{matrix}
-1  &  x & -1 &  0  &...&  0 \\
 0  & -1 &  x & -1  &...&  0 \\
 .  &    &    &  .  &   &  . \\
 .  &    &    &  .  &   &  . \\
 .  &    &    &  -1 &  x& -1 \\
 0  &    &    &     & -1& x  \\
 -1 &    &    &     &  0&-1  \\   
\end{matrix} \right|_{(N-1)\times(N-1)}
\end{equation*}
We calculate the determinants $D^*$ and $D^{**}$ by minors according to the last row to get 
\begin{equation*}
D^* = (-1)(-1)^{2N-2}D_{N-2}+(-1)(-1)^{N-1+1}(-1)^{N-2}=(-1)^{2N-1}[D_{N-2}+1]
\end{equation*}

\begin{equation*}
D^{**}=(-1)(-1)^{2N-2}(-1)^{N-2} +(-1)(-1)^{1+N-1}D_{N-2}=(-1)^{3N-3}+(-1)^{N+1}D_{N-2}
\end{equation*}
Therefore, 
\begin{eqnarray*}
D_N & = & xD_{N-1} + (-1)^{2N-1}[D_{N-2}+1]+(-1)^{N+2}[(-1)^{3N-3}+(-1)^{N+1}D_{N-2}]\\
    & = & xD_{N-1} + (-1)^{2N-1}[D_{N-2}+1]+(-1)^{4N-1}+(-1)^{2N+3}D_{N-2}\\
    & = & xD_{N-1} + D_{N-2}[(-1)^{2N-1}+(-1)^{2N+3}]+(-1)^{2N-1}+(-1)^{4N-1}\\
    & = & xD_{N-1} - 2D_{N-2}-2
\end{eqnarray*}
Substituting the solution for $D_{N-1}$ and $D_{N-2}$, we get 
\begin{equation*}
D_N = \frac{x\sin(N\theta)-2\sin((N-1)\theta)-2\sin(\theta)}{\sin(\theta)}=0
\end{equation*}

Further simplification of $D_N$ leads to 
\begin{eqnarray*}
D_N &=& 2\frac{\cos(\theta)\sin(N\theta)-[\sin((N-1)\theta)+\sin(\theta)]}{\sin(\theta)}\\
    &=& \frac{2}{\sin(\theta)}[\cos(\theta)\sin(N\theta)-[2\sin(\frac{N\theta}{2})\cos(\frac{(N-2)\theta}{2})]\\
    &=& \frac{4}{\sin(\theta)}[\cos(\theta)\sin(\frac{N\theta}{2})\cos(\frac{N\theta}{2})-\sin(\frac{N\theta}{2})[\cos(\frac{N\theta}{2})\cos(\theta)+\sin(\frac{N\theta}{2})\sin(\theta)]\\
    &=& -4\sin^2(\frac{N\theta}{2})
\end{eqnarray*}

Equating $D_N=0$ we get the solutions
\begin{equation*}
\theta_p=\frac{2\pi p}{N}
\end{equation*}
for $p=0,1,...,(N-1)$

Therefore, the eigenvalues of the Rouse ring are 
\begin{equation*}
\lambda_p=2(1-cos(\theta_p))= 2(1-cos(2\pi p/N))
\end{equation*}
Since $\cos(\frac{2\pi \phi}{N})=\cos(\frac{2\pi(N-\phi)}{N})$, we have eigenvalues multiplicities. In our case, setting $\phi=p$ we have $\cos(\frac{2\pi p}{N})=\cos(\frac{2\pi(N-p)}{N})$ for $p=1,2,3,..,(N-2)$. For the end-values, $p=0$ and $p=N-1$ the eigenvalues are unique. 

\section{The Pair Correlation function}\label{thePairCorrelationFunction}
According to \cite{doi1996introduction}. To investigate the spatial distribution of segments in the Rouse polymer we introduce the segment pair correlation function, $g(r)$, defined as follows. If we focus on the $n^{th}$ segment, we define $g_n(r)$ to be the average segment density at position $r$ from segment $n$. If we write $R_n(n=1,2,3,..,N)$ for the position vectors of the segments , then we can express $g_n(r)$ as 
\begin{equation*}
g_n(r)=\sum_{m=1}^N \left<\delta(r-(R_m-R_n)) \right>
\end{equation*} 

the average is a spatial average, so in 3D we have to divide by the volume $r+\delta r$ to get the concentration. The time correlation function, $g(r)$  is then defined as the average of the segment pair correlation function, $g_n(r)$. 
\begin{equation*}
g(r) = \frac{1}{N}\sum_{n=1}^{N}g_n(r)
\end{equation*}
In addition, we define $g(q)$, the Fourier transform of $g(r)$
\begin{equation*}
g(q) = \int e^{iqr}g(r)dr = \frac{1}{N}\sum\limits_{n=1}^{N}\sum\limits_{m=1}^{N}\left<exp([iq(R_m-R_n]) \right>
\end{equation*}
The pair correlation function is thus also called the radial distribution function. In the simplest terms it is a measure of probability of finding a particle at a distance $r$ away from a given reference particle. 


\section{Radius of Gyration}\label{radiusOfGyration}
From the definition of the Fourier transform of the pair correlation function for small $q$, we can define the length called radius of gyration. 

\section{Encounter probability in the Rouse ring}\label{section_encounterProbabilityInTheRouseRing}
Following the method in \cite{amitai2012computation}. First we transform the Rouse ring system of equation into normal coordinates using the Rouse ring eigenvalues found in Section \ref{section_eigenvaluesOfTheRouseRing}. We transform to normal coordinates by $u_p=\sum_{n=1}^N \alpha_p^nR_n$, where $\alpha_p$ are the eigenvectors. We then represent our boundary conditions that bead $j$ and bead $k$ are in the same position, using the normal coordinates representation. The probability density function in the normal coordinates should then satisfy the forward Fokker Planck equation. The solution to the probability density function should then be expanded as an infinite sum of the form 
\begin{equation*}
p(x,t)=\sum_{i=1}^\infty a_iw_{\lambda_i^\epsilon}(x)e^{-\lambda_i^\epsilon tD}e^{-\phi(x)}
\end{equation*}
where $w_{\lambda_i^\epsilon}(x)$ and $\lambda_i^\epsilon$ are the eigenvectors and eigenvalues of the forward Fokker Planck operator outside of the domain, and $a_i$ are coefficients. Integrate the probability density according to $x$, over the domain without the absorbing boundary (the encounter sphere), to get a function of time. An integration of the resulting function from time $t=0$ to $\infty$ is the first moment of the survival function, which is the mean first encounter time (i.e the first time the polymer's ends are at the absorbing sphere). 


\section{Center of mass diffusion of the rouse ring}\label{section_centerOfMassDiffusionOfTheRouseRing}
By summing the rows of the Rouse ring matrix, we see that the differential equation describing the diffusion of the center of mass does not change in relation to the chain's organization. Inner loops in the ring do not change the center of mass diffusion. Therefore 
\begin{equation*}
\frac{dcm}{dt} = \frac{1}{N}\sum_{n=1}^N g_n
\end{equation*}
we follow calculation by 
[Unfinished]

\section{Simulations}
\subsection{Choosing $\Delta t$}
The dynamics of beads is governed by the equation
\begin{equation*}
\frac{dP}{dt} = -\frac{3D}{b^2}RP+\sqrt{2D}f_n
\end{equation*}

if we represent the Rouse matrix in its diagonal form 
\begin{equation*}
R=Q^{-1}\Lambda Q
\end{equation*}
where $Q$ is an orthonormal matrix with the eigenvectors of $R$ in its columns,and $\Lambda$ is a diagonal matrix with the Rouse eigenvalues in the diagonal, we get 
\begin{equation*}
\frac{dP}{dt} = -\frac{3D}{b^2}Q^{-1}\Lambda Q+\sqrt{2D}f
\end{equation*}
multiplying from the left by $Q$ and setting $S=QP$ we get 

\begin{equation*}
\frac{dS}{dt}=-\frac{3D}{b^2}\Lambda S+\sqrt{2D}Qf
\end{equation*}
\subsubsection{The case with no noise}
if we omit the noise term and solve for $S$ 
the numerical scheme is 
\begin{equation*}
S(t+\Delta t)-S(t)=(-\frac{3D\Delta t}{b^2}\Lambda)S(t)
\end{equation*}
taking the norm of both sides and dividing 
\begin{equation*}
\frac{\|S(t+\Delta t)-S(t)\|}{\|S(t)\|}=\|(-\frac{3D\Delta t}{b^2}\Lambda)\| = \frac{3D\Delta t}{b^2}\lambda_{max}= \frac{12D\Delta t}{b^2}
\end{equation*}
where for the Rouse matrix $\lambda_{max} = 4$.  Demanding that the quotient be smaller than 1 we have 
\begin{equation*}
 \frac{12D}{b^2}\leq1 \Longrightarrow \Delta t \leq \frac{b^2}{12D}
 \end{equation*}
 \subsubsection{The case with noise}
 using the same procedure as above, 
 \begin{equation*}
\frac{\|S(t+\Delta t)-S(t)\|}{\|S(t)\|}=\|(-\frac{3D}{b^2}\Lambda)S+\sqrt{2D}Qf\|/\|S\|\leq(\frac{3D}{b^2}\|\Lambda)\| +\sqrt{2D}\|Qf\|/\|S\|
 \end{equation*}
 \begin{equation*}
 =\leq \frac{3D}{b^2}\lambda_{max} +\sqrt{2D}\frac{\|Q\|\|f\|}{\|S\|}=\frac{12D}{b^2}\leq\frac{3D}{b^2}\lambda_{max}\Delta t +\sqrt{2D}\Delta t
 \end{equation*}
Demanding that the quotient be smaller than 1 we have 
\begin{equation*}
 \frac{12D}{b^2}\leq1 \Longrightarrow \Delta t \leq \frac{b^2}{12D+\sqrt{2D}b^2}
 \end{equation*}
\subsection{Relaxation time}
The relaxation time is determined according to the formula
\begin{equation}
\tau_p = \frac{b^2\Delta t}{12d^2\sin^2(\frac{p\pi}{2N})}
\end{equation} 

% The bibliography
\bibliographystyle{plain}
\bibliography{TheoryOfPolymerDynamicsBibliography} % the bibliography.bib file 
\end{document}



